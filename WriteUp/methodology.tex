\chapter{Methodology}

For my project I used a mix of Waterfall, KanBan and Rapid Application Development approaches. I used these approaches as I found I could quickly iterate development cycles using key features of these approaches without hindering application process too far. I Believe this approach to be valid, as I was quickly able to see merit in my developed application and showed iterative progress throughout. 

\section{Waterfall}

For my project I used a General Waterfall approach which follows a linear, sequential approach to software development, requiring that each stage be finished before moving on to the next. The reason it is called a "waterfall" development is because it transitions smoothly from one phase to the next.\cite{al2017project}

The Waterfall model, a traditional method for developing software, is still employed in some circumstances when the project's requirements and scope are clear. It has been criticised, though, for being rigid and for not allowing for adjustments or modifications throughout the development process.\cite{al2017project} However Seeing as some of my requirements were bound to change I decided to employ a personal approach to KanBan boards and Rapid Application Development.

\section{KanBan}

As work items move through different stages of a process, they are managed and tracked using a KanBan board, a visual project management tool. It was first used in the manufacturing sector to streamline production procedures, but software development teams have since adopted it to control their workflows.\cite{patel2018kanban}
A KanBan board, in its most basic form, consists of a board with columns labelled "To Do," "In Progress," and "Done," which represent the various stages of the development process. As tasks advance through the various stages of the process, represented by cards or sticky notes, they are moved across the board. The board offers a visual representation of the status of every task and the project's overall development.\cite{pressman2015software}

The software development lifecycle can be managed and tracked using kanban boards, which give teams the ability to see their workflow, spot bottlenecks, and continuously improve their procedures.\cite{pressman2015software} Fortunately, I was developing this application alone so I was able to construct my boards without the use any third party applications.

\section{Rapid Application Development}

Iterative development and rapid prototyping are key components of the rapid application development (RAD) software development methodology. By concentrating on delivering functional software as soon as possible and improving it through feedback and collaboration, it seeks to shorten the time and cost of software development.\cite{al2017project}
In RAD, the software development process is divided into smaller, easier-to-manage stages that are then iteratively developed and tested. In order to deliver a high-quality product more quickly, the development team collaborates closely with stakeholders to gather feedback and make continuous improvements to the software.\cite{pressman2015software}

\section{Combination}

For my approach I designed the underlying system and how the user should interact with the system. Each task was su divided into smaller i.e. User can login and connect to network, User can view current active lobbies, User can join lobbies. Along with this I used GitHub to keep a ledger of tasks completed unfortunately my naming convention wasn't exactly strict and my commit messages didn't fully encapsulate certain tasks which ended up causing confusion on what tasks were appropriately completed leading to further problems down the line.

While developing I continued to use my favourite approach to software development, rapid application development. Which is mainly a prototyping methodology but I have found it incredibly useful when developing and deploying multiplayer network systems fast, so that I can test without compromise as my main focus was to make sure players could connect with one another.

At the start of this year, I had convened with my supervisor on a few occasions to discuss the project when it was under a different idea and structure. Unfortunately, I started having difficulties communicating with supervisor making a disconnect between myself and the supervisor. This disconnect severly hindered the development of the project as I began having doubts about the project discussed which eventually led to me changing the project idea altogether in order for an easier development cycle, this in fact further threw the project into turmoil as I no longer had any structure to follow, what didn't help either was that I convinced myself that after the disconnect I could not talk to said supervisor, which led to further issues.



